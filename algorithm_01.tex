\chapter{Garment Segmentation}
\label{garment_segmentation}
\comment{
This chapter is devoted to the first part of our algorithm, extracting the garment data from the sensor data. Data is obtained as a point cloud, and then converted to a depth image. Then, garment segmentation is perfomed.
}

\section{Garment segmentation}
\label{garment_segmentation_mask}

The RGB-D image obtained in the previous step contains both the clothing article and the table on which it rests. Therefore, after retrieving the data, a segmentation process is required to classify whether a pixel represents the garment or the table.

For this purpose, many methods could have been chosen, based on both color and depth information. But, as the main focus of our work is unfolding clothes, a simple color-based method was selected. 

We work under the assumption that the garment has been placed over a flat clear surface, as opposed to the garment which is much more colorful (saturated). First, the RGB image was converted to the HSV space. Working in the HSV space gives us direct information about our magnitudes of interest: saturation and intensity. As preprocessing, to reduce the effect of the noise on the segmentation process, a convolution with a 5x5 Gaussian kernel with $\sigma=1.1$ is computed on the saturation (S) and value (V) channels of the HSV image.

A thresholding operation is then applied to the filtered image, using Otsu's algorithm \comment{(ref, maybe?)} to obtain the optimal threshold values. Pixels with low amount of saturation, and a high values are classified as being part of the table, as opposed to dark or saturated pixels.

Finally, some morphological transformations are applied to the resulting mask to reduce noise due to false positives/negatives. A 5x5 square kernel is used in several closing operations, followed by a similar number of opening operations.


\begin{figure}[thpb]
    \centering
    \includegraphics[width=0.48
    \textwidth]{figures/placeholder.png}
    \caption{\comment{Again, I should put here a picture of the resulting mask}}
    \label{fig:segmentation_mask}
\end{figure}

\section{Contour extraction}
From the mask obtained in the previous step (section \ref{garment_segmentation_mask}) a blob labeling algorithm is applied to detect the garment outline. This outline will be used in later steps to obtain the candidates to be a fold.

The contour extraction method used is the Topological Analysis by Border Following algorithm developed by Suzuki and Abe\comment{[ref to suzuki85]}. This is a widely used algorithm for connected-component labeling and countour finding. Only external contours were retrieved. A simple chain approximation was then applied to reduce the number of points that describe the contours, storing only the endpoints of the different segments.

Due to noise, sometimes some small blobs appear in segmentation masks, so the extracted contour with the highest area was selected as garment. This way, those small blobs were discarded.
After obtaining the garment contour, it is processed, as we want to obtain a further simplified garment outline. We assume the fold line has a very high probability of lying in the garment outline and, therefore, this contour will represent all the candidate segments to be a fold. 

To obtain the simplified outline, the Ramer–Douglas–Peucker algorithm \comment{[ref to article]} is applied. This algorithm recursively divides the contour in segments by choosing the first and last points of the curve and drawing a line. Then, it checks whether that point is closer to that line than a threshold $\epsilon > 0$ or not. If it is closer, all points not marked to be kept can be discarded, otherwise, if it is greater than $\epsilon$, that point is marked to be kept and the procedure is repeated cosidering the last marked point as ending point. If there are no points left at one stage, the last point of the contour becomes the ending point again. The previous ending point becomes then the new starting point.

The parameter $\epsilon$ is calculated from the magnitude of the contour perimeter, considering it to be 1\% of that value. The greater this value is, the more simplified the resulting contour will be.