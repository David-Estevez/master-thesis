\chapter{Conclusions and future work}
\label{conclusions_and_future_work}

This chapter will describe the main contributions made to the state of the art by this work. Along the development of this work and during the experiments some challenges appeared. They will be covered after the main contributions along with future work lines that address those challenges and  further improve the state of the art.


\section {Main contributions}
\label{conclusions:contributions}

This section is dedicated to explain the main contributions of this work, and develop them in detail according to the different steps in our unfolding algorithm.

The main contribution of our work is that its analysis of the garment is not dependent of a prior model. The process core resides in the segmentation of the grayscale image representing the heights. Results show that our approach is promising to be included in a complete pipeline of clothes folding.

\begin{enumerate}
	\item \textbf{Garment Segmentation:} Regarding garment segmentation, our main contribution is a segmentation stage that is independent of the shape and color of the garment. This stage works as long as the table that holds the garment is white or grey and the garment is more colorful than the table. The segmentation stage, additionally, requires no user input to label background/foreground samples.
	\item \textbf{Garment Depth Map Clustering:} Our algorithm improves existing approaches by using Watershed as clustering algoritm, so that a threshold value for labelling contiguous regions is not required. This absence of threshold value also avoids breaking similar height regions into different label regions due to large wrinkles. 
	\item \textbf{Garment Pick and Place Points:} The main contribution of this work regarding   choosing pick and place points is that the selection of those points is independent of the garment category. Previous work found in the literature required the garment category or model to be specified or learnt to have a prior knowledge of the most suitable grasping points.
\end{enumerate}

\section{Future Work}
\label{conclusions:future_work}
Some opportunities exist to further develop and improve this work, despite the presented satisfactory results. This section will introduce those current issues and possible solutions to address them, regarding each of the stages that compose our algorithm.

\begin{enumerate}
	\item \textbf{Garment Segmentation:} As our garment segmentation stage currently depends on color, it sometimes trims some areas of the garment due to ilumination or a wrong white balance of the RGB camera. This segmentation problem may result in an incorrect polygon approximation that affects the functionality of the system.
	\item \textbf{Garment Depth Map Clustering:} In general, we have found that working with a single point of view is a very limiting factor, as occlusions sometimes make folds ambiguous. As our approach does not depend on garment models, dissambiguating those situations is very challenging. For that reason, the author strongly believes that moving to an approach that uses a 3D point cloud of the garment as input data would have more information available to solve those ambiguous situations in a better way.

	\item \textbf{Garment Pick and Place Points:} While the selected points may serve as a rule-of-thumb for robotic system developers, there is no clear metric for evaluating cuantitatively the suitability of the selected points. Therefore, experimental validation through further experiments with robotic systems is required to determine whether a better pick and place strategy exists:
	
\begin{itemize}
\item For instance, the highest garment point or highest region centroid could be used as alternative to the current pick point, \comment{under the assumption that that point would correspond to an overlapped garment region not attached to the garment regions underneath}.

\item Other place points could also be chosen depending on a different criteria. For example, the place point could be calculated using the fold line as axis of symmetry, or determined based on the unfolding trajectory selected.	

\item The trajectory used to unfold the garment is a simple one, which follows a straight line connecting the pick point, a point above the pick point, a point above the place point and the place point. More elaborated trajectories, such as splines or curves, could be used instead, such the one Li et al. present in their method for folding deformable objects \cite{Li2015IROS}.

\end{itemize}
\end{enumerate}

Finally, as the experiments were performed using data gathered from a limited set of garments in our laboratory environment, the statistical relevance of the results is also limited. Future works should include the use of datasets of garments shared and used among other laboratories and research groups to promote fair benchmarking through repeatable experiments.
