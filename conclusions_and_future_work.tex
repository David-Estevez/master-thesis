\chapter{Conclusions and future work}
\label{conclusions_and_future_work}

This chapter will describe the main contributions made to the state of the art by this work, as well as the challenges that appeared along the development of this work and during the experiments. Finally, future work lines to address those challenges and to improve further the state of the art will be proposed.

\comment{Results show that our approach is promising to be included in a complete pipeline of clothes folding.}
 
 
\section {Main contributions}
\label{conclusions:contributions}

The main contribution of our work is its analysis of the garment not dependent of a prior model. The process core resides in the segmentation of the grayscale image representing the heights.

Though the results are acceptable, two aspects of the algorithm can be improved.
Contour detection currently trims some areas of the garment, which at times results in an incorrect polygon approximation that affects the functionality of the system.
The garment pick point, the unfolding direction, and the place point have been chosen in a somewhat arbitrary manner. While the selected points may serve as a rule-of-thumb for robotic system developers, further experiments with robotic systems are required to determine whether a better pick and place strategy exists. For instance, the highest garment point or highest region centroid could be used as alternatives to the current pick point.
% which can be considered the optimal pick and place strategy. 
%It is clear through experiments presented (and not) in the paper that two aspects of the algorithm can be perfected: 
%Future work 
%shold focus 
%- Extracción de contorno
%- Forma de desdoblar
%Other segmentation algorithms (such as Simple Linear Iterative Clustering) have been tested, with unsatisfactory results. However, we are confident that other segmentation algorithms may improve the results.

\section {Possible improvements}
\label{conclusions:improvements}

\section{Future Work}
\label{conclusions:future_work}
One of the short-term future works is to implement a method to select more than one unfolding direction. Currently, the direction with the smallest bumpiness value is selected. In some cases, several directions share a very small value. This fact make us think that an interpolation between these directions may improve the final result. 