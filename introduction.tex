\chapter{Introduction}
\label{introduction}

Folding clothes is a common and necessary, but tedious, task for humans. Additionally, due to the increasing aging of the world population, a growing need exists for 
\juansays{automated solutions}
to be able to help us with laundry.
\juansays{(justificar uso de robot: complexidad tarea y la environment)}

\begin{figure}[htbp]
    \centering
    \includegraphics[width=0.8\textwidth]{figures/placeholder2.png}
    \caption{\comment{Current clothes folding solutions}}
    \label{setup}
\end{figure}

Working with non-rigid objects such as clothes is a difficult task for robots, due to the complexity of modeling and manipulating deformable, thin objects. Clothes can be easily entangled when doing laundry, and recognizing individual garments and their category just from color or depth image analysis becomes an almost impossible task, due to occlusions amongst the cluttered clothes. Another challenging aspect when working with deformable objects is how to bring the object into a known configuration from an arbitrary initial state.

Extensive work can be found in literature about automated clothes folding once the garment category has been identified  \reftodo, as well as for modeling the garment for fold/wrinkle removal or selecting the most suitable grasping point/strategy \comment{(section \ref{state_of_the_art})}. For this reason, this work focuses on how to unfold a clothing article that has been picked up from a pile of clothes and is placed on a flat surface.

It is assumed that a clothing article has already been separated from the rest of the clothes to be folded, and placed on a flat surface. The garment could have been placed on that surface either by a robot or by a human coworker, allowing a collaborative folding pipeline in which a human and a robot can perform different parts of the folding process.
As our algorithm is not based on a geometrical model of the garment to be unfolded, it is general enough to be used with any category of garment, from towels and blankets to trousers or shirts, and with any number of folds. 
%
The presented approach consists in using a depth image from a single point of view to find regions of the garment overlapping other regions, which are considered to be folds. Then, all the possible candidate paths are studied to determine the unfolding direction. This is an iterative process to be repeated until the garment is fully unfolded.

\section{Objectives}
\label{intro_objectives}
The main objective pursued in this work is to develop an algorithm that can estimate the grasping and release points for a deformable object so that a manipulator robot can iteratively unfold a garment for determining its garment category and the folding sequence to apply.

From the aforehead mentioned algorithm we can deduce the following specific objects:

\begin{itemize}
	\item It should rely as little as possible in color or patterns present in the garment, to be independent from the illumination.
	\item It should provide a general method of detecting folds in deformable objects without a prior model of the garment to be unfolded.
	\item It should be able to estimate the best position of the grasping point, direction of movement, and release point in order to unfold the detected fold.
\end{itemize}

\juansays{estaría bien algo de texto aquí}

\section{Structure}
\label{intro_structure}

The structure of this document is the following. \juansays{coletilla}

\begin{itemize}
\item Chapter \ref{state_of_the_art} provides an overview of the current state of the different methods and techniques to achieve \comment{robot cloth folding}.
\item In chapter \ref{architecture}, the actual algorithm is presented. 
\item Chapter \ref{experiments_and_results} is dedicated to introduce the experimental setup, describe the experiments performed and analyze the results obtained.
\item Finally, some conclusions and lines of future work are presented in chapter \ref{conclusions_and_future_work}.

\end{itemize}
