En la actualidad, los métodos para el doblado automático de ropa usando robots requieren una vista completa de la prenda extendida, para su clasificación y posterior doblado basada en un modelo de la categoría a la que pertenece la prenda. En esta tesis, se presenta un algoritmo independiente del tipo de prenda que no requiere de un modelo previo para desdoblar ropa y que está basado en el uso de una única vista obtenida con un sensor RGB-D. Una vez desdoblada, se puede aplicar para su doblado cualquier algoritmo ya existente.

El algoritmo presentado en este trabajo está divido en 3 etapas principales. Primero, una Etapa de Segmentación separa la información de la prenda de la del fondo, y aproxima su contorno a un polígono. Después, una Etapa de Agrupación encuentra regiones de altura similar en la prenda, correspondientes a las distintas partes solapadas. Finalmente, una Etapa de Puntos de Agarre y Posicionamiento encuentra los puntos más adecuados para sujetar y soltar la prenda durante el proceso de desdoble, basados en un valor de \textit{agrura}, definido como la diferencia de alturas acumulada a lo largo de las trayectorias de desdoble candidatas.

La evaluación del algoritmo se llevó a cabo a través de experimentos con un conjunto de datos que comprende 120 muestras de 6 categorías de prenda distintas, con uno y dos dobleces. Los resultados fueron analizados, y presenta puntuaciones altas para cada una de las etapas que componen el algoritmo. El algoritmo de desdoble ha sido validado también a través de experimentos llevados a cabo con un robot humanoide.
